\documentclass[a4paper,10pt,dvipdfmx]{jsarticle}

\usepackage{latexsym,amsmath,amssymb} % 数式関連. とりあえず入れとく 
\usepackage{graphicx} % includegraphics に必要
\usepackage{algorithm}
\usepackage{algpseudocode}

\algrenewcommand\algorithmicrequire{\textbf{入力:}} 
\algrenewcommand\algorithmicensure{\textbf{出力:}}


\begin{document}

\begin{algorithm}[htbp]
    \caption{ユークリッドの互除法}
    \label{algo1}
    \begin{algorithmic}[1]
    \Require 自然数$a,b \in \mathbb{Z_+} \ \backslash\  \{0\}$
    \Ensure $a,b$の最大公約数
    \State $c \leftarrow a\%b$; 
    \While {$c \neq 0$}
        \State  $a \leftarrow b$;
        \State  $b \leftarrow c$;
        \State  $c \leftarrow a\%b$;
    \EndWhile{;}
    \State 現在の$b$を, 元の$a,b$の最大公約数として出力
    \end{algorithmic}
\end{algorithm}


\begin{algorithm}[htbp]
    \caption{挿入ソート}
    \label{algo2}
    \begin{algorithmic}[1]
    \Require $n$個の実数$A[0],A[1],\dots,A[n-1] \in \mathbb{R}$
    \Ensure $A[0],A[1],\dots,A[n-1]$を昇順に整列した系列
    \For {$i \leftarrow 1,2,\dots,n-1$}
        \State  $j \leftarrow i-1$;
        \State  \textit{key} $ \leftarrow A[i]$;
        \While{$A[j] > $ \textit{key}}
            \State $A[j+1] \leftarrow A[j]$;
            \State $j \leftarrow f-1$
            \If{$j = -1 $}
                \State break
            \EndIf
        \EndWhile{;}
        \State $A[j+1] \leftarrow $ \textit{key}
    \EndFor{;}
    \State $A[0],A[1],\dots,A[n-1]$を出力
    \end{algorithmic}
\end{algorithm}


% \ref{algo1}
% \ref{algo2}

\begin{thebibliography}{100}
    \bibitem{} 最適花子, 数理太郎, 最適化入門, 大手出版社, 2021. 
    \bibitem{} 数理太郎,数理二郎.〇〇問題に対する分枝限定法.某学会論文誌,\textbf{8}, pp.~23-34~(2019).
    \bibitem{} T. Yamada, N. Aoki and M. Marukami. Which criteria evaluate baseball players most appropriately? \textit{Journal of Sport Managiment}, \textbf{23}, pp.~123-456~(2012).
\end{thebibliography}

\end{document}


