\documentclass[a4paper,12pt,dvipdfmx]{article}
%\documentclass[a4paper,10pt]{article}
%\jtwocolumn


\pagestyle{myheadings}\markboth{Classifier Criteria: \today}{Classifier Criteria: \today}

%%%%%%%%%%%%%%%%%%%%%%%%%%%%%%%%%%%%%%%%%%%%%%%%%%%%%%%%%%%%%%%%%%%%%%%%%%%%%%
% packages
\usepackage{graphicx}
\usepackage{epstopdf}
%\usepackage{tabularx}
\usepackage{latexsym}
%\usepackage{pstricks,psfrag}
\usepackage{amsmath,amssymb}
%\usepackage{algorithm} %%% for comm. to N-sensei
%\usepackage{algpseudocode} %%% for comm. to N-sensei
\usepackage[ruled,vlined,linesnumbered]{algorithm2e}
\usepackage{tikz}

%%%%%%%%%%%%%%%%%%%%%%%%%%%%%%%%%%%%%%%%%%%%%%%%%%%%%%%%%%%%%%%%%%%%%%%%%%%%%%
% newcommand
%\renewcommand{\mid}{:~}
\newcommand{\secref}[1]{Section~\ref{sec:#1}}
\newcommand{\tabref}[1]{Table~\ref{tab:#1}}
\newcommand{\figref}[1]{Figure~\ref{fig:#1}}
\newcommand{\corref}[1]{Corollary~\ref{cor:#1}}
\newcommand{\lemref}[1]{Lemma~\ref{lem:#1}}
\newcommand{\propref}[1]{Proposition~\ref{prop:#1}}
\newcommand{\thmref}[1]{Theorem~\ref{thm:#1}}

\newcommand{\vc}[1]{\mbox{\boldmath$ #1 $}}
\newcommand{\myhline}{
  \noindent
  \begin{tabular}{p{0.97\textwidth}}
    \\\hline
\end{tabular}}
\newcommand{\tab}{\hspace*{1em}}



%%%%%%%%%%%%%%%%%%%%%%%%%%%%%%%%%%%%%%%%%%%%%%%%%%%%%%%%%%%%%%%%%%%%%%%%%%%%%%
% newtheorem
\newtheorem{cor}{Corollary}
\newtheorem{lem}{Lemma}
\newtheorem{prop}{Proposition}
\newtheorem{thm}{Theorem}

%%%%%%%%%%%%%%%%%%%%%%%%%%%%%%%%%%%%%%%%%%%%%%%%%%%%%%%%%%%%%%%%%%%%%%%%%%%%%%
% newenvironment
\newenvironment{proof}{\medskip
  \noindent{\scshape Proof:}}{\quad $\Box$\medskip}

\newenvironment{myframe}{\begin{trivlist}\item[]
    \hrule
    \hbox to \linewidth\bgroup
    \advance\linewidth by -30pt
    \hsize=\linewidth
    \vrule\hfill
    \vbox\bgroup
    \vskip15pt
    \def\thempfootnote{\arabic{mpfootnote}}
    \begin{minipage}{\linewidth}}{%
    \end{minipage}\vskip15pt
    \egroup\hfill\vrule
    \egroup\hrule
\end{trivlist}}

\long\def\invis#1{}

%%%%%%%%%%%%%%%%%%%%%%%%%%%%%%%%%%%%%%%%%%%%%%%%%%%%%%%%%%%%%%%%%%%%%%%%%%%%%%
% style definitions
%
% following setting makes 3cm spaces for top and bottom, and
% 2.5cm spaces for left and right
%
% default setting
\setlength{\oddsidemargin}{22pt}         % 62pt
\setlength{\evensidemargin}{22pt}        % 62pt
\setlength{\headheight}{12pt}            % 12pt
\setlength{\textheight}{662pt}           % 592pt
\setlength{\marginparsep}{10pt}          % 10pt
\setlength{\footskip}{30pt}              % 30pt
\setlength{\hoffset}{-13pt}              % 0pt
\setlength{\paperwidth}{597pt}           % 597pt
\setlength{\topmargin}{20pt}             % 20pt
\setlength{\headsep}{25pt}               % 25pt
\setlength{\textwidth}{427pt}            % 327pt
\setlength{\marginparwidth}{106pt}       % 106pt
\setlength{\marginparpush}{5pt}          % 5pt
\setlength{\voffset}{-37pt}              % 0pt
\setlength{\paperheight}{845pt}          % 845pt


% 1 inch = 2.54 cm = 72.27 pt

\renewcommand{\baselinestretch}{1.20}
% \setlength{\columnsep}{1.0cm}

%%%%%%%%%%%%%%%%%%%%%%%%%%%%%%%%%%%%%%%%%%%%%%%%%%%%%%%%%%%%%%%%%%%%%%%%%%%%%%
% title information
\title{}
\author{}
\date{}

%%%%%%%%%%%%%%%%%%%%%%%%%%%%%%%%%%%%%%%%%%%%%%%%%%%%%%%%%%%%%%%%%%%%%%%%%%%%%%
% document
\begin{document}
\section{Criteria of Learning Performance of A Classier}

Suppose that we are given a data set $D=\{(x_1,a_1),(x_2,a_2),\dots,(x_m,a_m)\}$ such that $x_i\in\mathbb{R}^K$ and $a_i\in\{0,1\}$, $i=1,2,\dots,m$,
where we call $a_i$ the {\em class of $x_i$}.
A feature vector $x_i$ with $a_i=1$ (resp., 0) is called
{\em positive} (resp., negative). 
We call a function $\eta:\mathbb{R}^K\to\{0,1\}$
that estimates the class of a feature vector a {\em classifier}. 

There are various criteria to evaluate the learning performance of a classifier.
We summarize some of them. 

\subsection{Accuracy Based Criteria}
Let us partition the data set $D$
into $D=D_1\cup D_0$,
where $D_1$ (resp., $D_0$)
is the set of all positive (resp., negative) feature vectors in $D$. 
%
For a classifier $\eta$,
a feature vector $x_i\in D$ is called
\begin{itemize}
\item {\em true positive} if $a_i=1$ and $\eta(x_i)=1$;
\item {\em true negative} if $a_i=0$ and $\eta(x_i)=0$;
\item {\em false positive} if $a_i=0$ and $\eta(x_i)=1$; and
\item {\em false negative} if $a_i=1$ and $\eta(x_i)=0$. 
\end{itemize}
We denote by $\textrm{TP}(\eta;D)$/$\textrm{TN}(\eta;D)$/$\textrm{FP}(\eta;D)$/$\textrm{FN}(\eta;D)$
the sets of true positive/true negative/false positive/false negative
feature vectors, respectively.
We define $\textrm{TPR}(\eta;D)$,
$\textrm{TNR}(\eta;D)$,
$\textrm{FPR}(\eta;D)$ and $\textrm{FNR}(\eta;D)$ as follows;
\begin{align*}
  &\textrm{TPR}(\eta;D)\triangleq\frac{\textrm{TP}(\eta;D)}{|D_1|};\ \ 
  \textrm{TNR}(\eta;D)\triangleq\frac{\textrm{TN}(\eta;D)}{|D_0|};\\
  &\textrm{FPR}(\eta;D)\triangleq\frac{\textrm{FP}(\eta;D)}{|D_0|};\ \ 
  \textrm{FNR}(\eta;D)\triangleq\frac{\textrm{FN}(\eta;D)}{|D_1|}.
\end{align*}
It holds that $\textrm{TPR}(\eta;D)+\textrm{FNR}(\eta;D)=\textrm{TNR}(\eta;D)+\textrm{FPR}(\eta;D)=1$. 
%
The {\em accuracy} $\textrm{ACC}(\eta;D)$ is defined to be:
\[
\textrm{ACC}(\eta;D)\triangleq\frac{\textrm{TP}(\eta;D)+\textrm{TN}(\eta;D)}{|D|}.
\]
The {\em balanced accuracy} $\textrm{B-ACC}(\eta;D)$ is defined to be:
\[
\textrm{B-ACC}(\eta;D)\triangleq\frac{1}{2}(\textrm{TPR}(\eta;D)+\textrm{TNR}(\eta;D)).
\]

\subsection{ROC Curve and AUC}
Let $f:\mathbb{R}^K\to\mathbb{R}$ be a function and $\theta\in\mathbb{R}$ be a real number.
We construct a classifier $\eta_{f,\theta}:\mathbb{R}^K\to\{0,1\}$ as follows; for $x\in\mathbb{R}^K$,
\[
\eta_{f,\theta}(x)\triangleq\left\{
\begin{array}{ll}
  1 & \textrm{if\ }f(x)\ge\theta,\\
  0 & \textrm{otherwise}.
\end{array}
\right.
\]
Suppose that $x_1,x_2,\dots,x_m$ are sorted
so that $f(x_1)\le f(x_2)\le\dots\le f(x_m)$ holds. 
For $i=1,\dots,m-1$, let us define
\[
\theta_i\triangleq\frac{f(x_i)+f(x_{i+1})}{2},
\]
where we let $\theta_0\triangleq f(x_1)-\varepsilon$ and
$\theta_m\triangleq f(x_m)+\varepsilon$ for a small constant $\varepsilon\in\mathbb{R}_+\setminus\{0\}$.

We have $m+1$ classifiers $\eta_{f,\theta_0},\eta_{f,\theta_1},\dots,\eta_{f,\theta_m}$. Let us denote a 2D point $p_i\triangleq(\textrm{FPR}(\eta_{f,\theta_i};D),\textrm{TPR}(\eta_{f,\theta_i};D))$, $i=0,1,\dots,m$. 
Observe that $p_0=(1,1)$ holds since $\eta_{f,\theta_0}(x)=1$ holds
for all $x\in D$,  and that
$p_m=(0,0)$ holds since $\eta_{f,\theta_m}(x)=0$ holds
for all $x\in D$.
Also we have;
\begin{align*}
  &\textrm{FPR}(\eta_{f,\theta_m};D)=0\le\textrm{FPR}(\eta_{f,\theta_{m-1}};D) \le\dots\le\textrm{FPR}(\eta_{f,\theta_0};D)=1;\\
  &\textrm{TPR}(\eta_{f,\theta_m};D)=0\le\textrm{TPR}(\eta_{f,\theta_{m-1}};D) \le\dots\le\textrm{TPR}(\eta_{f,\theta_0};D)=1.
\end{align*}
The {\em Receiver Operating Characteristic curve} ({\em ROC curve}) {\em of $f$}
is a set of $m$ line segments $(p_m=(0,0),p_{m-1}),(p_{m-1},p_{m-2}),\dots,(p_1,p_0=(1,1))$.
The {\em Area Under Curve} ({\em AUC}) {\em of $f$},
which we denote by $\textrm{AUC}(f;D)$,
is defined to be the area between the ROC curve and the x-axis.
Hence we have $0\le\textrm{AUC}(f;D)\le1$. 

\begin{lem}
  \label{lem:auc}
  Suppose that we are given a data set $D=\{(x_1,a_1),(x_2,a_2),\dots,(x_m,a_m)\}$ such that $x_i\in\mathbb{R}^K$ and $a_i\in\{0,1\}$, $i=1,2,\dots,m$, and a function $f:\mathbb{R}^K\to R$, where $R\subseteq\mathbb{R}$.
  If $R=\{0,1\}$, then
  it holds that $\textrm{AUC}(f;D)=\textrm{B-ACC}(f;D)$. 
\end{lem}
\begin{proof}
  We see that the ROC of $f$ consists of
  three points, that is $(0,0)$, $(\mathrm{FPR}(f;D),\mathrm{TPR}(f;D))$ and $(1,1)$.
  Let $p:=\mathrm{FPR}(f;D)=1-\mathrm{TNR}(f;D)$
  and $q:=\mathrm{TPR}(f;D)$.
  Then we have
  \begin{align*}
    \textrm{AUC}(f;D) &= \frac{1}{2}pq + \frac{1}{2}(q+1)(1-p)\\
    &=\frac{1}{2}(pq+q+1-pq-p)\\
    &=\frac{1}{2}(1-p+q)\\
    &=\frac{1}{2}(\mathrm{TNR}(f;D) + \mathrm{TPR}(f;D))\\
    &=\textrm{B-ACC}(f;D). 
  \end{align*}
  \hfill
\end{proof}

\figref{auc} illustrates the ROC of $f$
in \lemref{auc}. 

\begin{figure}[h]
  \label{fig:auc}
  \centering
  \begin{tikzpicture}
  \label{roc}
  % 軸
  \draw[-stealth](-3,-3)--(-3,6)node[left]{TPR};
  \draw[-stealth](-3,-3)--(6,-3)node[below]{FPR};
  \draw(-3,-3)node[below left]{O};
  
  % 直線
  \draw (-3,-3)--(0,1.5);
  \draw (0,1.5)--(3,3);

  % 点線
  \draw[dashed](-3,3)--(3,3);
  \draw[dashed](3,3)--(3,-3);
  
  % 点線 (交点)
  \draw[dashed](-3,1.5)--(0,1.5);
  \draw[dashed](0,1.5)--(0,-3);


  % ラベル
  \draw(-3,3)--(-3,3)node[left]{$1$};
  \draw(3,-3)--(3,-3)node[below]{$1$};
  
  \draw(-3,1.5)--(-3,1.5)node[left]{$q$};
  \draw(0,-3)--(0,-3)node[below]{$p$};

  \end{tikzpicture}
  \end{figure}




\end{document}

